\documentclass{article}
\begin{document}
	Filip Jędrzejewski
	
	\section{Metoda Elementów Skończonych}
	
	\subsection{Problem obliczeniowy}
	
	\begin{equation}
		\frac{d^2\phi}{dx^2} = -\frac{\rho}{\varepsilon_r}
	\end{equation}
	
	przy czym:
	\begin{equation}
		\rho = 1
	\end{equation}
	
	\begin{equation}
		\varepsilon_r(x) =\left\{ \begin{array}{rl}
					10 & \textrm{dla $x \in [0,1]$} \\ 
					5 & \textrm{dla $x \in (1,2]$} \\
					1 & \textrm{dla $x \in (2,3]$}
					\end{array} \right.
	\end{equation}
	
	\subsection{Warunki brzegowe}
	Warunek brzegowy Cauchy'ego:
	\begin{equation}
		\phi '(0)+\phi (0) = 5
	\end{equation}
	 Warunek brzegowy Dirichleta:
	 \begin{equation}
	 	\phi (3) = 2
	 \end{equation}
	 
	 
	\section{Sformułowanie wariacyjne}
	Mnożymy równanie (1) przez funkcję testową $v$ spełniającą warunek: $v(3) = 0$:
	\begin{equation}
		\phi '' v = -\frac{\rho}{\varepsilon_r}v
	\end{equation}
	Całkujemy obustronnie:
	\begin{equation}
		\int_0^3 \phi '' v dx = -\int_0^3 \frac{\rho}{\varepsilon_r} v dx
	\end{equation}
	Całkując lewą stronę równania przez części otrzymujemy:
	\begin{equation}
		\phi ' (3) \cdot v(3) - \phi ' (0) \cdot v(0) - \int_0^3 \phi ' v ' dx = -\int_0^3 \frac{\rho}{\varepsilon_r} v dx
	\end{equation}
	Z własności funkcji $v(x)$ wiemy, że $v(3)=0$, zatem:
	\begin{equation}
		\phi' (0) \cdot v(0) + \int_0^3 \phi ' v ' dx = \int_0^3 \frac{\rho}{\varepsilon_r} v dx
	\end{equation}
	Z (4) mamy:
	\begin{equation}
		\phi ' (0) = 5 - \phi (0)
	\end{equation}
	Łącząc (9) i (10) otrzymujemy:
	\begin{equation}
		5 \cdot v(0) - \phi (0) \cdot v(0) + \int_0^3 \phi ' v' dx = \int_0^3 \frac{\rho}{\varepsilon_r} v dx
	\end{equation}
	\begin{equation}
		\int_0^3 \phi ' v' dx - \phi (0) \cdot v(0) = \int_0^3 \frac{\rho}{\varepsilon_r} v dx - 5 \cdot v(0)
	\end{equation}
	Równanie (12) można zapisać w następujący sposób:
	\begin{equation}
		B(\phi, v) = L(v)
	\end{equation}
	przy czym:
	\begin{equation}
		B(\phi, v) = \int_0^3 \phi ' v' dx - \phi (0) \cdot v(0)
	\end{equation}
	\begin{equation}
		L(v) = \int_0^3 \frac{\rho}{\varepsilon_r} v dx - 5 \cdot v(0)
	\end{equation}
	Zapiszmy:
	\begin{equation}
		\phi = w + \Phi
	\end{equation}
	Funkcja $w(x)$ spełnia warunek $w(3)=0$, zatem $\Phi(3) = 2$. Przyjmijmy zatem, że:
	\begin{equation}
		\Phi (x) = 2
	\end{equation}
	Łącząc zależności (13) i (17) otrzymujemy:
	\begin{equation}
		B(w+\Phi, v) = L(v)
	\end{equation}
	Korzystając z liniowości $B$ względem pierwszego argumentu, dostajemy:
	\begin{equation}
		B(w,v) + B(\Phi, v) = L(v)
	\end{equation}
	Zatem ostatecznie:
	\begin{equation}
		B(w,v) = L(v)-B(\Phi, v)
	\end{equation}
	
	
	
	
	
	
	
	
	
	
	
	
	
	
	
	
	
	
	
	
	
	
	
	
	
	
	
	
	
\end{document}